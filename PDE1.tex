%% LyX 2.0.5.1 created this file.  For more info, see http://www.lyx.org/.
%% Do not edit unless you really know what you are doing.
\documentclass[11pt,english,ngerman]{scrreprt}
\usepackage[T1]{fontenc}
\usepackage[utf8]{inputenc}
\usepackage[a4paper]{geometry}
\geometry{verbose,tmargin=2.6cm,bmargin=3.5cm,lmargin=2.6cm,rmargin=2.6cm}
\usepackage{fancyhdr}
\pagestyle{fancy}
\setlength{\parskip}{\smallskipamount}
\setlength{\parindent}{0pt}
\usepackage{babel}
\usepackage{textcomp}
\usepackage{url}
\usepackage{amsmath}
\usepackage{amssymb}
\usepackage{graphicx}
\usepackage[unicode=true,
 bookmarks=true,bookmarksnumbered=true,bookmarksopen=false,
 breaklinks=true,pdfborder={0 0 0},backref=page,colorlinks=false]
 {hyperref}
\hypersetup{pdftitle={Partielle Differentialgleichungen I},
 pdfauthor={Andreas Völklein},
 pdfkeywords={Partielle Differentialgleichungen, Mathematik}}

\makeatletter

%%%%%%%%%%%%%%%%%%%%%%%%%%%%%% LyX specific LaTeX commands.
\providecommand{\LyX}{\texorpdfstring%
  {L\kern-.1667em\lower.25em\hbox{Y}\kern-.125emX\@}
  {LyX}}
\newcommand{\noun}[1]{\textsc{#1}}

%%%%%%%%%%%%%%%%%%%%%%%%%%%%%% Textclass specific LaTeX commands.
\usepackage{enumitem}		% customizable list environments
\newlength{\lyxlabelwidth}      % auxiliary length 

\@ifundefined{date}{}{\date{}}
%%%%%%%%%%%%%%%%%%%%%%%%%%%%%% User specified LaTeX commands.
\usepackage{tikz,pgfplots}
%\usepackage{tikz-3dplot,cancel,polynom}
\usetikzlibrary{matrix,arrows,calc,decorations.pathmorphing,intersections,shapes}
\usetikzlibrary{external}
\tikzexternalize
\usepackage{latexsym,stmaryrd,stackrel,braket,bbm,subfig,framed,esvect,scrhack,calc}
\usepackage [OMLmathrm,OMLmathbf,sfdefault=fav]{isomath}
\usepackage[explicit]{titlesec}
\usepackage[activate]{pdfcprot}

\pgfkeys{/pgf/number format/dec sep={\text{,}}}
\pgfplotsset{compat=newest}

% Inhaltsverzeichnis
\usepackage[subfigure]{tocloft}

\tocloftpagestyle{fancy}

\renewcommand{\cftchapindent}{1 em}
\renewcommand{\cftchapnumwidth}{1.5 em}

\renewcommand{\cftsecindent}{2.7 em}
\renewcommand{\cftsecnumwidth}{2.5em}

\renewcommand{\cftsubsecindent}{5.2 em}
\renewcommand{\cftsubsecnumwidth}{3.8 em}

\renewcommand{\cftsubsubsecindent}{9 em}
\renewcommand{\cftsubsubsecnumwidth}{4.5 em}

% Mathe-Operatoren
\DeclareMathOperator*{\exsop}{\exists}
\DeclareMathOperator*{\exsgop}{\exists!}
\DeclareMathOperator*{\fallop}{\forall}
\DeclareMathOperator*{\bcupdop}{\dot{\bigcup}}
\DeclareMathOperator*{\bcapdop}{\dot{\bigcap}}

%Operatornorm
\newcommand{\opnor}[1]{\abs{\hspace*{-1.1pt}\norm{#1}\hspace*{-1.1pt}}}

% nicht-totales Differential
\newcommand{\dBar}{\mathchar'26\mkern-12mu \textnormal{d}}

% Angström
\newcommand{\ang}{\textup{\AA}}

% schöne Vektorpfeile
\renewcommand{\vec}[1]{\vv{#1}}

% Rotieren
\newcommand{\Rotate}[1]{
\tikzset{external/export next=false}
\begin{tikzpicture}
\node[rotate=90] {\ensuremath{#1}};
\end{tikzpicture}
}

%QED-Zeichen (Box)
\newcommand{\qed}{\ensuremath{\Box}}
\newcommand{\qqed}[1][\arabic{chapter}.\arabic{section}\ifnum\arabic{subsection}>0{.\arabic{subsection}}\fi]{\hspace*{1mm}\hfill\qed\ensuremath{_{\text{#1}}}}

% Mengen Modulo
\newcommand{\moduloT}[2]{
\mbox{\raisebox{0.6ex}{\ensuremath{\displaystyle #1}}
{\hspace*{-1.5mm}\Large /}
\raisebox{-0.6ex}{\hspace*{-1.5mm}\ensuremath{\displaystyle #2}}
}}

% Links Modulo
\newcommand{\lmoduloT}[2]{
\mbox{\raisebox{-0.6ex}{\ensuremath{\displaystyle #1}}
{\hspace*{-1.5mm}\Large \ensuremath{\backslash}}
\raisebox{0.6ex}{\hspace*{-1.5mm}\ensuremath{\displaystyle #2}}
}}

% Für Z/2Z, um nicht soviel schreiben zu müssen
\newcommand{\modloT}[2]{\moduloT{ \mathbb{#1}}{#2\mathbb{#1}}}

%Laplace-Beltrami-Operator
\newcommand{\LBO}{
\begin{minipage}{6mm}
 \tikzset{external/export next=false}
 \begin{tikzpicture}
   \node at (0,0){$\Delta$};
   \draw[line width=0.75] (0.25,-0.13) -- (0.1,0.15);
 \end{tikzpicture}
\end{minipage}
}

%Die Modulo-Kommandos in klein, für die Darstellungen unter Quantoren.
\newcommand{\moduloScriptT}[2]{
\mbox{\raisebox{0.4ex}{\scriptsize\ensuremath{\displaystyle #1}}
{\hspace*{-1.5mm}\footnotesize /}
\raisebox{-0.4ex}{\hspace*{-1.5mm}\scriptsize\ensuremath{\displaystyle #2}}
}}

\newcommand{\lmoduloScriptT}[2]{
\mbox{\raisebox{-0.4ex}{\scriptsize\ensuremath{\displaystyle #1}}
{\hspace*{-1.5mm}\footnotesize \ensuremath{\backslash}}
\raisebox{0.4ex}{\hspace*{-1.5mm}\scriptsize\ensuremath{\displaystyle #2}}
}}

\newcommand{\modloScriptT}[2]{\moduloScriptT{ \mathbb{#1}}{#2\mathbb{#1}}}

% stehendes Winkelzeichen
\newcommand{\winkel}{
\tikzset{external/export next=false}
\begin{tikzpicture}[scale=0.25]
\draw ({-2+3^(1/2)},0) -- (0,1) -- ({2-3^(1/2)},0);
\draw ($(0,1) + ({cos(235)*0.7},{sin(315)*0.7})$) arc (235:315:0.7);
\end{tikzpicture}}

% Wurzel mit Häkchen
\newcommand{\hsqrt}[2][{}]{\setbox0=\hbox{$\sqrt[#1]{\phantom{|}\!\! #2\hspace*{1pt}}$}\dimen0=\ht0
  \advance\dimen0-0.2\ht0
  \setbox2=\hbox{\vrule height\ht0 depth -\dimen0}
  {\box0\lower0.4pt\box2}}

% Damit nicht immer "Kapitel 1" etc. über der Kapitelüberschrift steht
\titleformat{\chapter}
  {\huge\bfseries}
  {\textrm{\thechapter} }{0pt}
  {\textrm{#1} \thispagestyle{fancy}
  }

% Neudefinition der Abschnittsmarker für die Kopfzeile
\renewcommand\partmark[1]{\markboth{#1}{}}
\renewcommand\chaptermark[1]{\markright{\arabic{chapter} #1}}
\renewcommand\sectionmark[1]{}
\renewcommand\subsectionmark[1]{}

% Schriften auf Serif umstellen
\addtokomafont{descriptionlabel}{\rmfamily}
\addtokomafont{disposition}{\rmfamily}

% Zeilenumbrüche in Gleichungen
 \allowdisplaybreaks

% Kopf- und Fußzeile
% Höhe der Kopfzeile
\setlength{\headheight}{14pt}
% obere Trennlinie
%\renewcommand{\headrulewidth}{0.4pt}
\fancyhf{} %alle Kopf- und Fußzeilenfelder bereinigen
\fancyhead[L]{\textbf{Partielle Differentialgleichungen I}} %Kopfzeile links
%\fancyhead[C]{\leftmark} %zentrierte Kopfzeile
\fancyhead[R]{\rightmark} %Kopfzeile rechts
\fancyfoot[C]{\thepage\quad\!\!\!\slash\quad\!\!\!\pageref{END-front}} %Seitenzahl der Front-Matter

\AtBeginDocument{
  \def\labelitemi{\normalfont\bfseries{--}}
  \def\labelitemii{\(\circ\)}
  \def\labelitemiii{\(\triangleright\)}
}

\makeatother

\begin{document}






\global\long\def\norm#1{\left\lVert #1\right\rVert }


\global\long\def\abs#1{\left\lvert #1\right\rvert }


\global\long\def\opnorm#1{\opnor{#1}}


\global\long\def\BRA#1{\Bra{#1}}


\global\long\def\KET#1{\Ket{#1}}


\global\long\def\BraKet#1{\Braket{#1}}


\global\long\def\mins{\textnormal{-}}


\global\long\def\LB{\LBO}


\global\long\def\exs{\exsop}


\global\long\def\exsg{\exsgop}


\global\long\def\fall{\fallop}


\global\long\def\bcupd{\bcupdop}


\global\long\def\bcapd{\bcapdop}


\global\long\def\sr#1#2#3{\underset{#3}{\overset{#2}{#1}}}


\global\long\def\dd{\textnormal{d}}


\global\long\def\DD{\textnormal{D}}


\global\long\def\dbar{\dBar}


\global\long\def\angs{\ang}


\global\long\def\TT{\textnormal{T}}


\global\long\def\ii{\textbf{i}}


\global\long\def\modulo#1#2{\moduloT{#1}{#2}}


\global\long\def\lmodulo#1#2{\lmoduloT{#1}{#2}}


\global\long\def\modlo#1#2{\modloT{#1}{#2}}


\global\long\def\moduloScript#1#2{\moduloScriptT{#1}{#2}}


\global\long\def\lmoduloScript#1#2{\lmoduloScriptT{#1}{#2}}


\global\long\def\modloScript#1#2{\modloScriptT{#1}{#2}}


\global\long\def\vek#1{\vectorsym{#1}}


\global\long\def\mat#1{\matrixsym{#1}}


\global\long\def\ten#1{\tensorsym{#1}}


\global\long\def\msd#1{\mathstrut_{#1}}


\global\long\def\msu#1{\mathstrut^{#1}}


\pagenumbering{roman}


\title{\hspace*{1mm}\vspace*{-15mm}\\
{\Huge Partielle Differentialgleichungen I}}


\author{\vspace*{-5mm}\\
\textit{\small Vorlesung von}\\
\textit{\noun{\small Prof. Dr. Felix Finster}}\\
\textit{\small im Sommersemester 2013}\\
\textit{\small Überarbeitung und Textsatz in \LyX{} von}\\
\textit{\noun{\small Andreas Völklein}}\\
\vspace*{5mm}\\
\includegraphics[clip,width=15cm]{unir}\\
\vspace*{3mm}\\
{\normalsize Stand: \today}\\
\vspace*{-30mm}}

\maketitle
\fancyhead[R]{Lizenz}


\subsubsection*{ACHTUNG}

Diese Mitschrift ersetzt \emph{nicht} die Vorlesung.

Es wird daher \emph{dringend} empfohlen, die Vorlesung zu besuchen.

\vfill{}


\selectlanguage{english}%

\subsubsection*{Copyright Notice}

Copyright © 2013 \noun{Andreas Völklein}

Permission is granted to copy, distribute and/or modify this document
under the terms of the GNU Free Documentation License, Version 1.3
or any later version published by the Free Software Foundation;

with no Invariant Sections, no Front-Cover Texts, and no Back-Cover
Texts.

A copy of the license is included in the document entitled “GFDL”.


\subsubsection*{Disclaimer of Warranty}

\noun{Unless otherwise mutually agreed to by the parties in writing
and to the extent not prohibited by applicable law, }\textbf{\noun{the
Copyright Holders and any other party, who may distribute the Document
as permitted above,   provide the Document “as is}}\textbf{”,}\textbf{\noun{
without warranty of any kind}}\noun{, expressed, implied, statutory
or otherwise, including, but not limited to, the implied warranties
of merchantability, fitness for a particular purpose, non-infringement,
the absence of latent or other defects, accuracy, or the absence of
errors, whether or not discoverable.}


\subsubsection*{Limitation of Liability}

\textbf{\noun{In no event}}\noun{ unless required by applicable law
or agreed to in writing }\textbf{\noun{will the Copyright Holders,
or any other party, who may distribute the Document as permitted above,
be liable to you for any damages}}\noun{, including, but not limited
to, any general, special, incidental, consequential, punitive or exemplary
damages, however caused, regardless of the theory of liability, arising
out of or related to this license or any use of or inability to use
the Document, even if they have been advised of the possibility of
such damages.}

\textbf{\noun{In no event will the Copyright Holders'/Distributor's
liability to you}}\noun{, whether in contract, tort (including negligence),
or otherwise, }\textbf{\noun{exceed the amount you paid the Copyright
Holders/Distributor}}\noun{ for the document under this agreement.}

\selectlanguage{ngerman}%

\subsubsection*{Links}

Der Text der „\foreignlanguage{english}{GNU Free Documentation License}“
kann auch auf der Seite
\begin{quote}
\url{https://www.gnu.org/licenses/fdl-1.3.de.html}
\end{quote}
nachgelesen werden.

Eine transparente Kopie der aktuellen Version dieses Dokuments kann
von
\begin{quote}
\url{https://github.com/andiv/pde1}
\end{quote}
heruntergeladen werden.

\newpage{}

\fancyhead[R]{Literatur}


\subsection*{Literatur}
\begin{itemize}
\item \noun{Name des Autors: }\emph{Titel des Buches}; Verlag, <Erscheinungsjahr>\\
ISBN: 978-1-2345-6789-0
\end{itemize}
{\small \newpage{}}\fancyhead[R]{Inhaltsverzeichnis}
\fancyhead[C]{}

\tableofcontents{}\label{END-front}\newpage{}\pagenumbering{arabic}
\fancyfoot[C]{\thepage\quad\!\!\!\slash\quad\!\!\!\pageref{END}} % Seitenzahl des Hauptteils
\fancyhead[R]{\rightmark}
%\fancyhead[C]{\leftmark}%DATE: Mi 17.04.2013


\chapter{erstes Kapitel}


\section{erster Abschnitt}


\subsection{erster Unterabschnitt}


\part*{Anhang\thispagestyle{empty}}

\addcontentsline{toc}{part}{Anhang}

\fancyhead[R]{Index}
\fancyhead[C]{Anhang}


\chapter*{Danksagungen}

\addcontentsline{toc}{section}{\hspace*{2.7em}Danksagungen}

\fancyhead[R]{Danksagungen}

Mein besonderer Dank geht an Professor Finster, der diese Vorlesung
hielt und es mir gestattete, diese Vorlesungsmitschrift zu veröffentlichen.

Außerdem möchte ich mich ganz herzlich bei allen bedanken, die durch
aufmerksames Lesen Fehler gefunden und mir diese mitgeteilt haben.

\vspace{1cm}


\hfill{}Andreas Völklein

\label{END}
\end{document}
